\section{Literature Review}\label{sec:scholarly}

\subsection{Supply-Side Arguments of Gendered Occupational Segregation}

Over the past two decades, the body of sociological and social-psychological literature characterizing supply-side mechanisms for gendered occupational segregation has progressed significantly. In contrast to demand-side arguments, which examine the effect of discrimination and bias at the institutional and interactional level, supply-side arguments characterize the forces which ultimately lead to the ``subconscious gendering of men’s and women’s self-expressive career decisions'' \cite{cech_self-expressive_2013}. The focus, then, is not on how applicants are recruited and evaluated, but rather on the processes funneling individuals into gender-typed degree programs, social organizations, and applicant pools, and the mechanisms by which these processes are disguised as ``self-expression.'' (It is worth noting that arguments for supply-side and demand-side dynamics are not necessarily in opposition, but instead come together to collectively contribute to more diversified inequality regimes.) A key process noted in the literature on supply-side dynamics is the role of status generalizations for self-perceptions of competence and belonging; these works argue that gendered cultural beliefs inform the extent and manner in which individuals incorporate feedback into evaluations of their own ability and fit in gendered occupational fields. Specifically, I trace the literature relevant to supply-side mechanisms affecting gender minorities in technical fields in the last two decades, which collectively shows that gender minorities are subject to greater marginalization and negative feedback in their educational paths to technical fields than straight cis-men, and further, that they are more likely to incorporate this negative feedback into their self-perceptions, resulting in greater attrition from these educational sequences (and by extension inadequate representation in the applicant pool.)  
	
Correll argued in 2001 that, controlling for actual ability, individuals’ self assessments of their competence in gendered career-relevant tasks are influenced by cultural beliefs about gender. Specifically, this longitudinal observational study showed that eighth-grade through college-aged women perceived themselves as less competent at math (argued to be a highly masculine-typed skillset) and more readily incorporated feedback (in the form of standardized test-taking performance) into their self-perceptions of their mathematical ability than men. These effects did not manifest identically for verbal skills, instead showing that men rated themselves as similarly competent on average, yet were more prone to incorporate negative feedback into their self-perceptions of their own ability in this area, showing that this phenomenon is generalizable beyond women in masculine-typed fields. More generally, then, ``the appraisals individuals make of their own competence at various tasks are more contingent on local evidence when societal expectations for success are lacking.'' Further, in this study, Correll finds that more positive self-assessments result in greater chances of persisting on a given career path, even after controlling for ability (as measured through test-taking performance.) These findings, taken together, articulate a supply-side mechanism for unequal representation in the candidate field for gendered occupations \cite{correll_gender_2001}.   

In 2004, Correll furthered her argument on this mechanism by implementing it in an experimental setting. After exposing undergraduate subjects to beliefs that men were more competent at a given task, she found that the men subjects assessed their own ability more highly than women and were more likely to express interest in careers for which men were supposedly more fit for. These effects disappeared when subjects were instead exposed to egalitarian views. Through the use of experimental methodology, Correll further supports her earlier argument that people’s self-perceptions of their abilities in gender-typed skillsets are influenced by cultural gender beliefs and, further, that individuals more readily incorporate feedback into their self-perceptions when they confirm beliefs perceived to be widely held about gendered task competence. These studies were foundational for developing understandings of the way that ``status generalization occurs in individual evaluative settings'' in that self-evaluation is indeed social to the extent that people utilize status characteristics to inform their relative position in some greater collective \cite{correll_constraints_2004}.  

Hargittai and Shafer implement a similar experimental approach in a 2006 paper on web-use skills. In this study, the authors do not find differences between men and women in ability to navigate online content. However, women participants rated their own Web-use skills, defined as the ability to locate information and resources on the internet, significantly lower than men. The authors hypothesize that this difference in self-perception could lead to significant disparities in both absolute exposure to internet use as well as the kind of content that participants engage with. This research shows that the online setting itself is a gendered context in regard to self-evaluation, even without the presence of task assignments which are themselves gendered outside of an internet context \cite{hargittai_differences_2006}. While the task of basic information-finding on the internet might seem dated, these findings are particularly relevant to the practice of navigating technical documentation. Especially within the base-R environment, many resources and help-files, as well as the pathways to locate them, have not changed since the publishing of this study in 2006. Even in the context of a tidy based workflow, interaction with the R console and RStudio IDE in searching for relevant and informative documentation presents a task that is arguably more reminiscent of early internet resource-finding than modern Web-surfing.  

In 2011, Cech et al. furthered these understandings of how individual evaluations are informed by status generalization to beget a supply-side mechanism for gendered occupational segregation. The study found that, once engineering undergraduates selected their major, their self-perceptions of their own competence in engineering did not significantly influence their intent to remain in the field. The study also disconfirms previous hypotheses that women are more likely to consider family plans when choosing careers, thus resulting in aspirations for more female-typed occupations that are more likely to have normative cultures or significant policy centered on accomodating families. Instead, as the mechanism explaining how individuals choose—and decide whether to remain in—gendered occupations, the authors propose a mechanism that they name professional role confidence, focusing not only on individuals’ self-perceptions of actual competence, but also in their ability to fulfill roles and identity features associated with gendered occupations. Professional role confidence was shown to be especially pivotal for persistence in pursuit of gendered-occupations. Further, the authors argued that women have less professional role confidence in general and that this disparity is amplified in engineering and other masculine-typed occupations \cite{cech_professional_2011}.  

Another instance of this professional role confidence at play was articulated two years earlier by Cheryan et al. Again moving beyond self-evaluation of skillsets as a mechanism for generating career interest, Cheryan et al. argued a theory of \textit{ambient belonging} in 2009, in that individuals use cues from their physical environment to inform their professional role confidence. Focusing on women interested in computer science, the authors conduct a series of studies demonstrating that ``stereotypes can be communicated (and altered) merely through the physical cues present in an associated environment,'' and further that the demasculinization of the physical environment is ``sufficient to boost female undergraduates’ interest in computer science to the level of their male peers'' \cite{cheryan_2009}. This series of studies articulates another supply-side mechanism for gendered occupational choice, where gender minorities not only differentially incorporate feedback and evaluation into their self-perceptions of their belonging in a gendered occupation, but also gendered environmental cues.

In a series of recent papers, Cech and coauthors have extended gendered understandings of supply-side mechanisms beyond the cis-binary in studies of the experiences of gender minorities in undergraduate engineering programs and STEM-related government agencies. (My use of the term gender minorities here might be, in some ways, more evocative of gendered minorities; I include, in addition to people whose gender identity is inadequately represented in a given context, people who identify as lesbian, gay, bisexual, or any other non-heterosexual sexual orientation in acknowledgement that the lived experiences of people holding these identities differ from those of people identifying as straight in regard to the way that they are socialized as gendered individuals.) Initially, in an analysis of data from several federal agencies, the authors found that LGBT employees in governmental STEM organizations reported significantly lower job satisfaction and had higher turnover rates than their straight, cisgender colleagues \cite{cech_queer_2017}. More recently, Cech and Rothwell replicated this study with a larger sample across a greater number of federal agencies, this time exclusively those which have LBGT-inclusive policies. Results were consistent with previous research, though they incorporated an intersectional approach into their methods with racial and racial-interactive effects \cite{cech_lgbt_2019}. From a supply-side standpoint, these studies show that gender minorities are less likely to remain in STEM fields even after their entry into the workforce, exaggerating the already significant occupational segregation due to differences in career longevity. In 2018, Cech and Rothwell examined cultures of inclusivity—and their effects on individual self-perception—at eight U.S. engineering schools. The study found a persistent negative climate for gender minorities across each of the schools, arguing that this was indicative of educational institutions’ mirroring of predominant workplace cultures. This negative climate manifested both in the devaluation of work produced by gender minorities and in measures of social marginalization \cite{cech_lgbtq_2018}. This finding is especially consequential in light of Correll’s earlier works arguing that people differentially incorporate this negative feedback into their self-perceptions of their competence in accordance with dominant cultural gender beliefs. Doubly, under Cech’s framework of professional role confidence, even if gender minorities do not see this discriminatory treatment as reflective of their actual capabilities, their exposure to this climate could significantly affect their representation in the applicant pool through their perceptions of their ability to fulfill the roles and identity features associated with a given role-incumbent schema \cite{gorman_gender_2005}.  

Recent literature has demonstrated the continuing disadvantageous treatment that gender minorities face in undergraduate STEM education. For instance, in their 2017 paper, Blair et al. argue that the sampled STEM faculty predominantly characterize their gender conceptualizations as gender blind or gender acknowledging, both of which hold that ``interventionist positions to promote gender equity [are] inappropriate'' and places ``action to support gender equity outside the scope of their duties as faculty members.'' In refusing to engage in proactively disrupting bias in their curricular design, these faculty continue to engage in pedagogical practices catering to socialized hegemonic masculinity \cite{ohland_race_2011}. In 2015, Seron et al. argue that practices of professional socialization which are predominant in undergraduate engineering programs act as another mechanism for the reproduction of sex segregation. Specifically, the emphasis on collaborative work, though more reflective of professional engineering contexts, is argued to interact harmfully with gendered individual self-perceptions, as well as the differential incorporation of feedback confirming gendered cultural stereotypes, wherein women are made to explicitly negotiate ``assessing where one fits into the pecking order'' in men-dominated groups. Further, emphasized exposure to internship and shadowing experiences narrow gender minorities’ understanding of role-incumbent schemas in professional contexts, challenging their professional role confidence and, as I argued earlier, consequently decreasing their representation in these fields \cite{seron_persistence_2016}. Again, in light of Correll and Cech’s arguments described above, the findings of these studies are especially crucial for understanding retention of gender minorities in technical fields. Given that these populations differentially incorporate performance feedback and socialization into their self-perceptions of their own competence and belonging, and the demonstrated salience of these evaluations for career-path persistence, the (unsatisfied) need for inclusive STEM classroom cultures is made clear.  

The literature on supply-side mechanisms for gendered occupational segregation in the last two decades collectively articulates a powerful mechanism for the exclusion of gender minorities from STEM fields. Taken together, the outlined body of work shows that gender minorities continue to be subject to greater disadvantageous treatment throughout their educational paths to technical fields. Further, it is shown that this negative feedback is more likely to be incorporated into gender minorities’ self-evaluations of their ability and belonging, and that these accumulated socialized differences in self-perception measurably impact resilience in educational attainment.   

It is through this lens that I approach documentation practices in statistical computing. At the intersections of statistics and computer science, statistical computing is deeply embedded in—and draws methodological and cultural elements from—each of science, technology, engineering, and math. In this way, the social theory developing supply-side mechanisms for occupational gender segregation, and its application to understanding gender minorities’ representation in STEM fields, is relevant for understanding the representation of gender minorities in statistical computing.  

\subsection{R, RStudio, \& the tidyverse}\label{sec:tidy}

% The language
\textit{R} is an open-source programming language designed for statistical computing, and it is widely used across contexts ranging from academic research to business analytics to introductory statistics classes. At the time of writing, the first versions of the language were released over 25 years ago, and the language has been under active development since then. 

% CRAN
The stock capabilities of R are supplemented by a large body of user-created \textit{packages}, which are collections of code libraries (often containing \textit{functions}, which are code snippets that allow users to automate common tasks), datasets, and combinations of both. 

The founding of the \textit{Comprehensive R Archive Network (CRAN)} coincided with the first public beta release of the language in 1997. CRAN serves to facilitate this community development,  hosting packages in a common repository for all R users to freely download and modify. At the time of its initial release, CRAN hosted 12 packages; it now hosts over 15,000 \cite{cran}.

% RStudio
In 2011, an integrated development environment (IDE) called \textit{RStudio} was released for the language, providing users with a graphical interface to more easily interact with the language. The software is free for individual users, but commercial and server licenses are paid. A substantial proportion of R development occurs within RStudio, and one of the major R user conferences, \textit{rstudio::conf}, is hosted by the company.

% tidyverse
In 2014, Hadley Wickham, a prominent data scientist, formalized what he called \textit{tidy} data---tabular datasets ``where each column is a variable, each row is an observation, and each cell contains a single value'' \cite{wickham2014tidy}. Wickham developed and continues to develop a collection of packages built around this shared principle of data representation, providing tools to transform data into a tidy format, and further to extract meaning from data after it is ``tidied.'' The development of this collection of packages, which came to be known as the \textit{tidyverse}, now includes contributions from hundreds---if not thousands---of individual contributors of open-source code. Throughout this paper, I refer to tidy packages and non-tidy packages. In reality, this boundary is not obvious; while there is a concrete (yet adapting) set of packages formally included in the tidyverse, a substantial portion of tidyverse-unaffiliated package development still aligns to tidy data principles. I address the implications of this boundary work more thoroughly in Section \ref{sec:disc}.

% RStudio and tidyverse alignment, capitalism
Notably, though, the tidyverse is largely a product of RStudio. Dr. Wickham has been employed by the company since January 2013 \cite{wickhamhadley}, and the company employs several prominent R data scientists full-time who, almost exclusively, contribute open-source code to improve package functionality in---and to promote---the tidyverse \cite{rstudiopackages, future_r_hadley}.

% tidyverse and Inclusivity
The R community, especially the tidyverse, has been regarded as a much more inclusive and diverse group of users than that which utilizes competing data science toolkits such as Python---a 2015 study estimated that 9.2\% of R package authors are women, whereas 2\% of Python package authors are women \cite{mair2015motivation}. In a recent interview, Dr. Reshama Shaikh, a prominent data scientist and organizer for women in the field, said that, compared to the Python community, the R community is ``really relaxed, and fun, and welcoming'' \cite{shaikhwomen2019}. Similarly, in a recent interview, Dr. Wickham noted that ``people tell me they love R because the community is so welcoming. I think a large part of that is because of that it has become much more diverse'' \cite{future_r_hadley}. Dr. Shaikh notes that she believes an essential element of fostering this culture took place through coordinated, collective organization:

\begin{displayquote}
``Everyone is committed to diversity. Everybody wants inclusivity, and they're all working hard, and they all have initiatives, but when I looked at the Python community, it's like each group is in its own individual kayak and they're rowing hard, everyone's looking at the same direction and they're moving there and they're all working hard, but the R community has sort of invested in this really large boat and they can move to their goal a lot faster with a lot less rowers to get where they want to go'' \cite{shaikhwomen2019}.
\end{displayquote}

This community coordination is driven by two primary organizing forces: the R-Ladies organization and RStudio (and by extension, then, the tidyverse.) Founded in 2012, R-Ladies ``is a global grassroots organisation whose aim is to promote gender diversity in the R community'' \cite{grieves}. R-Ladies has had a profound and widespread impact on representation of gender minorities in the R user community, and its membership is indicative of gender representation by programming language; while it is estimated that there are more than six times as many users of Python than users of R, the R-ladies organization has almost 30,000 members, while PyLadies (the analogous gender diversity advocacy group for Python) has only 36,500 members \cite{shaikhblog}. Wickham recently reflected on RStudio's unique position as a for-profit company to make an impact on the community to which it markets to in a recent interview.

\begin{displayquote}
``There is also this broader question about how we make sustainable open-source software. Companies get this huge economic benefit from it, and they are not required to give back. It’s very hard to rely on philanthropy, so how do we extract some of the economic value open-source is generating and reinvest it back into the community?'' \cite{future_r_hadley}.
\end{displayquote}

Much of the reinvestment that Wickham mentions takes place through the R Consortium, a collection of organizations ``chartered to fund and inspire ideas that will enable R to become an even better platform for science, research, and industry'' \cite{rstudio_about}. In 2018, the R Consortium announced that R-Ladies would be a named a top-level project, providing the organization with ``long term investment for success'' \cite{mertic_r}. In this way, the tidyverse community's investment in increasing representation of gender minorities becomes clear.

With this context in mind, I now examine the degree to which these conceptualizations of representation are reflected through code documentation.




