\section{Introduction}\label{sec:intro}

Occupational sex segregation remains resilient in the United States despite the significant economic benefits shown to be associated with integration \cite{sex_seg}. Rather than attributing these gaps in representation completely to discrimination or coercion on the part of employers, recent scholarship also articulates supply-side mechanisms for occupational sex segregation, showing how ``the reproduction of occupational sex segregation [is in part due to] the deeply personal, self-reflective---yet culturally and structurally gendered---career decision making of men and women'' \cite{cech_self-expressive_2013}. Rather than regarding ``interest'' in careers as inherent to individuals, this literature demonstrates how self-expression is part and parcel of gendered socialization. In this paper, I will show how these mechanisms come to bear on individuals pursuing careers in statistical computing. Specifically, I examine two user communities formed around the R programming language, and how the representation of gender minorities in these communities is deeply interconnected with practices of code documentation prevalent among them. Rather than conceptualizing code documentation as unargumentative or ideologically neutral, I argue that these resources are capable of delivering feedback to readers about their competence and identity alignment with careers in computational statistics and, further, that this is especially consequential for readers whose identities do not align with their conceptualization of the typical worker.
