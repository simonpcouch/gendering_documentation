\section{Results}\label{sec:results}

\subsection{Qualitative Interviews}\label{sec:results-qual}

\subsubsection{Confounding Factors} \hspace{10pt} Throughout the course of interviewing, these developers mention several other mechanisms that might lead to differences in the appearance of documentation between tidyverse packages and other packages. I begin by noting each of these proposed mechanisms, regardless of the frequency with which they came up, in order to most generously acknowledge the role that other factors, some of which have been previously discussed, might play in driving this association between gender representation and documentation practices.

A principal mechanism that the interviewees suggest might be driving these differences in documentation practices between development communities are the increased resources available to the core developers of the tidyverse team, as compared to other open-source developers. Interviewee $A$ noted that ``100\% of their day, 100\% of their work is package development, so they spend a lot more time and a lot more energy than people who develop packages, but that’s not their only job.'' Given that the tidyverse team's principal work is open-source package development, they can devote more time and energy to the overall process than many others who write open-source software, say, after working some other (paid) development job. When this additional time is available, this interviewee suggests, it is more likely to be spent on documentation than building out code functionality. 

Both interviewees also pointed out that the tidyverse team is in much closer communication with each other than the R development community outside of the tidyverse. As interviewee $A$ noted, ``if there’s something that’s working for one package, there tends to be a decision to do the same thing for other packages as well.'' This uniformity in documentation structure and appearance within the tidyverse, then, could be a result of the close connections between developers in the tidyverse team, ``who are all working under the same roof,'' resulting in quicker dissemination of practices within the community than outside of it. Interviewee $B$ shared this sentiment, noting that, within the tidyverse team, ``there’s a strong culture of information sharing of good development practices.'' These structural influences, then, allow documentation within the tidyverse to share a common structure and quality.

At the same time, though, both interviewees also note that many developers outside of the tidyverse community make use of the many packages developed by the tidyverse team. In addition to writing tools for R users, the tidyverse team has developed many popular tools for R package developers, including the \textit{pkgdown}, \textit{usethis}, \textit{roxygen2}, and \textit{devtools} packages, each of which automates or simplifies development and documentation workflows. As interviewee $A$ noted, ``when a decision is made about ‘this is a good idea,’ that gets implemented in either \textit{pkgdown} or a package like \textit{usethis}... then it becomes a lot easier for others to adopt the same practices.'' In this way, documentation from developers outside of the tidyverse team comes to look more similar to documentation written by the tidyverse team. Similarly, as interviewee $B$ noted, ``What has made it easy for other people to adopt these practices [shared by the tidyverse team] is that then they built packages for making it easier'' to write documentation. In this way, the organizational influence on documentation practices is weakened, as developers outside of the core team still have access to the tools that make writing effective documentation easier. Interviewee $A$ also noted that R developers outside of the tidyverse team still ``probably take cues about writing documentation from Hadley’s \textit{R Packages},'' a book written the tidyverse founder Hadley Wickham that has become the gold standard manual for R package development. These cues come both in the form of the general approach and language that is exemplified in \textit{R Packages} as well as the many references to the aforementioned packages developed by the tidyverse team \cite{wickham2015r}.

An additional mechanism by which interviewee $A$ proposed that documentation practices might differ are the actual principles by which many tidyverse packages are written. While I propose that the effect predicted in Hypothesis $2(d)$ is reflective of more effective documentation practices, this interviewee also adds that many packages within the tidyverse contain functions that are more related to each other than functions provided by packages outside of the tidyverse. ``There probably are more linked functions within the tidyverse than in some other packages... tidyverse packages, from a design perspective, are designed in a way that'' the value of the package functionality lies more in the connections between functions than in other R packages. A forthcoming publication by the tidyverse team states that ``carefully designing functions so that the inputs and outputs align'' and ``identifying the atoms of a problem and the ways in which [their solutions] might be composed to solve bigger problems'' are underlying principles of the design philosophy of the tidyverse \cite{wickham2020tidyverse}. In this way, the greater interconnections between functions evoked in documentation within the tidyverse could be due to the design principles underlying the code itself.

\subsubsection{Supply-Side Mechanisms} \hspace{10pt} In addition to the alternative mechanisms for the association between gender representation and documentation practices (via the tidyverse) mentioned above, though, the interviewees elaborated on the differential effect of poorly written documentation for gender minorities, and the arguments for a relationship beyond correlation between representation of gender minorities in development communities and the quality of documentation resulting from those communities.

Both of these interviewees use the language of a ``user base'' when describing the implications for poorly written documentation. Their arguments presuppose a pipeline model, where gender minorities (and minorities in other identities) are better represented in a pool of potential developers, and ``gradually drop out'' as these learners move farther along in their educational and career paths. As interviewee $A$ said, ``I do believe that writing better documentation is helpful for capturing a greater percentage of your user base, which inevitably, hopefully, will mean you’re also capturing the people who are most likely to fall through the cracks at the first step.'' Interviewee $B$ conceptualized this process of leakage in a similar way: ``writing better documentation will inevitably increase your user base, and it is more likely to then, you know, capture certain subpopulations there too.'' Interviewee $A$ pointed to students' first interactions with R in introductory classes as this pivotal first step: ``If it’s a student [who is interacting with poorly written documentation], they’re not going to enjoy it, and then they’ll remember statistics as something annoying they had to do... You probably lost your customer, if you will.'' This same interviewee argued that these implications of documentation practices for inclusivity move beyond the classroom, saying that if these prospective users ``otherwise wouldn’t see themselves as stereotypical programmers... from a welcoming or inclusive perspective, again, it depends on whether they feel like `is this something I can master?' or not.'' After reflecting on literature in learning theory, she states, ``There has to be a relationship between an accessible nature of your documentation and whether or not someone who lands on your documentation page can think `yeah, this is something I can do.'\thinspace'' If the measures used in this study, then, are appropriate markers of accessibility, the mechanisms for greater representation of gender minorities within tidyverse development communities become clear.


Interviewee $A$ also reflected on how diversity improvement efforts are now structurally embedded in the production of documentation. Starting in early 2019, the tidyverse now hosts a ``tidyverse developer day'' (\textit{tidy-dev day}) at the conclusion of ``rstudio::conf'' and ``useR!'', two prominent R user conferences. With a stated goal to ``nurture regular contributors,'' the event provides a welcoming and inclusive environment for new contributors to the open-source tidyverse codebase, focusing on increasing diversity within the development community. As an advertisement for the first session promised, ``The tidyverse team will be there, so we can help you hit the ground running and/or get over any stumbling blocks that you encounter'' \cite{averick_tidyverse_2018}. This interviewee notes, too, ``When we’re going through and tagging issues as appropriate for tidy-dev day, there are a lot of issues that get tagged for documentation. This event also has a goal of increasing diversity, so in terms of who gets to attend it, there’s a deliberate effort to have a diverse crowd... Many of the issues they work on are documentation-related.'' In this way, representation of work done by contributors with minority identities within the tidyverse codebase becomes uniquely amplified in the production and revision of code documentation.

\subsection{Class of Object Analysis}\label{sec:results-coa}

In general, my analyses support the hypotheses presented in Section \ref{sec:met-analyze}.

As for package-level resources, I initially find that packages in the tidyverse are statistically significantly more likely (52\%) to provide vignettes in their documentation ($p = .005$). Further, packages from the tidyverse were 29\% more likely to provide master help-files than the sampled non-tidy packages, though this difference in proportions is not statistically significant, ($p = .143$). Thus, I find that Hypothesis 1 is generally supported (though only directionally for $1(b)$). 

My findings for function-level resources, in regard to hypotheses $2(a)$, $2(b)$, $2(c)$, and $2(d)$, are similar. On average, function help-files in the tidyverse contain 1468 more characters than those from non-tidy packages ($p < .001$). While the difference is not statistically significant, function help-files from tidy packages have, on average, .12 more examples than those from non-tidy packages ($p = .462$). In regard to comments, though, function help-files from the tidyverse have a statistically significantly greater mean number of comments ($\mu = 4.12$) than those from non-tidy packages ($\hat{\mu} = 1.73$, $p < .001$). Finally, the mean number of functions per help-file in the tidyverse ($\mu = 7.14$) is statistically significantly greater than that in non-tidy packages ($\hat{\mu} = 5.97, ~p < .001$). Thus, altogether, I find that Hypothesis 2 is supported (though only directionally for $2(b)$), and thus that help-files from functions in the tidyverse align with the proposed principles for effective and inclusive documentation more so than those from non-tidy packages.


