\section{Discussion}\label{sec:disc}

Future work on this topic can further the arguments made here in several ways. While experimentation evaluating the validity of this argument specifically may not be ethical, future work should strive to more thoroughly and rigorously evaluate code documentation practices as a causal mechanism for greater representation of gender minorities in computing. Namely, the differences in cultural and structural setting between the tidyverse and other R development communities go far beyond code documentation practices---demonstrating this same effect in a variety of computing contexts in which these same cultural and structural factors are reversed or not at play could strengthen the argument for this causal mechanism. Further, the line between tidyverse and non-tidyverse development is not an obvious one. I've referred to tidyverse packages as those included in the core tidyverse, as well as those refereed to on tidyverse publications as closely adjacent to the tidyverse. As the interviewees discussed, though, the tidyverse (as a toolkit) has a significant impact on the way that package development, and especially the writing of documentation, is carried out in other R development communities. Beyond those that just make use of tidyverse tools to write documentation, many package developers explicitly align with tidyverse principles in their development practices. In this way, alignment in development practices with the tidyverse exists on spectra far beyond a binary. Lastly, I note that the use of the term ``community'' in this paper is coarse, and may elicit the sense that these two development groups are similarly interconnected. In reality, the group developing the tidyverse is more often referred to as a ``team'' than a ``community'' by my interviewers, due to many of the group members' common professional affiliation with RStudio. On the other hand, those developing outside of the tidyverse framework make up a heterogeneous and decentralized group. Between these two groups, then, we see very different mediums for communication---while the tidyverse team is largely in frequent, direct communication, the primary mechanism for norm-setting of documentation practices outside of the tidyverse, if not making use of resources developed by the tidyverse, is likely mimicry of others' documentation.

Throughout this paper, I have argued that code documentation in the tidyverse is written more effectively and inclusively than other R code documentation, and that this is meaningful for understanding gender representation in the user community. Initially, I traced the literature articulating supply-side mechanisms for occupational segregation, showing that the work of gender minorities in STEM is subjected to greater scrutiny than their cisgender men counterparts. Further, too, gender minorities in these fields are more likely to incorporate this negative feedback, as well as their awareness of masculine elements of their physical surroundings, into their self-perceptions of their competence and identity alignment with the stereotypical worker. Collectively, these mechanisms coalesce into a system of exclusion discouraging gender minorities from pursuing careers in some STEM fields. I argued that these mechanisms are at play in the context of statistical computing, which incorporates methodological and cultural elements from many STEM fields and, further, that code documentation in statistical computing settings is capable of both delivering feedback about competence and transmitting norms about the stereotypical worker. Next, through interviews with women in these fields, I developed a set of measures to quantify the effectiveness and inclusiveness of code documentation, and analytically showed that code documentation from the tidyverse is more effectively and inclusively written than code outside of the tidyverse. These interviews also allowed me to more thoroughly scrutinize this association between gender minority representation in the tidyverse and the increased quality of documentation arising from the community. The interviewees articulated several confounding factors, as well as supply-side mechanisms, relating these two phenomena---namely, I argue that increased documentation effectiveness and inclusivity are a likely mechanism for the greater representation of gender minorities in tidyverse development communities, and also that, in turn, this greater representation contributes to more effectively and inclusively written documentation. That said, the confounding factors highlighted by the interviews are also plausible mechanisms driving this association, and the evidence provided is not sufficient to fully articulate a causal mechanism in either direction. 


\subsection*{Acknowledgements}

Many thanks to my two interviewees, whom graciously offered their time and perspective. Thank you to Dr. Kjersten Whittingon for her mentorship, advocacy, and valuable thoughts throughout the writing process. Lastly, many thanks to Dr. Kelly McConville and Dr. Andrew Bray for their insights on sampling strategies and inference. 